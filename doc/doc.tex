\documentclass[a4j]{jsarticle}

\usepackage{listings, jlisting}
\lstset{%
breaklines=true,
frame=single,
basicstyle=\scriptsize
}


\newcommand{\emptyline}{\vspace{1em}}


\begin{document}

\title{nan -novel analyzer-}
\author{@subaru45}
\date{2014-07-21}
\maketitle


\section{なんぞや}
nan は小説 (日本語) 解析ツールです。
以下の項目について調べてくれます:

\begin{itemize}
  \item 400字詰め原稿用紙 (20x20) 換算枚数
  \item 総文字数
  \item 行数
  \item 地の文/台詞比
  \item 漢字/ひらがな/カタカナ/その他比
  \item 行頭字下げをしているか?
  \item 括弧閉じ直前に句読点がきていないか?
  \item 「!」「?」の直後に全角スペースがあるか?
  \item 三点リーダ「…」は連続しているか?
  \item ダッシュ「―」は連続しているか?
\end{itemize}

つまり……

こんなテキストファイル
\begin{lstlisting}[caption=yodaka.txt]
 よだかは、実にみにくい鳥です。
 顔は、ところどころ、味噌(みそ)をつけたようにまだらで、くちばしは、ひらたくて、耳までさけています。
 足は、まるでよぼよぼで、一間(いっけん)とも歩けません。
 ほかの鳥は、もう、よだかの顔を見ただけでも、いやになってしまうという工合(ぐあい)でした。
 たとえば、ひばりも、あまり美しい鳥ではありませんが、よだかよりは、ずっと上だと思っていましたので、夕方など、よだかにあうと、さもさもいやそうに、しんねりと目をつぶりながら、首をそっ方(ぽ)へ向けるのでした。もっとちいさなおしゃべりの鳥などは、いつでもよだかのまっこうから悪口をしました。
「ヘン。又(また)出て来たね。まあ、あのざまをごらん。ほんとうに、鳥の仲間のつらよごしだよ。」
「ね、まあ、あのくちのおおきいことさ。きっと、かえるの親類か何かなんだよ。」
 こんな調子です。おお、よだかでないただのたかならば、こんな生(なま)はんかのちいさい鳥は、もう名前を聞いただけでも、ぶるぶるふるえて、顔色を変えて、からだをちぢめて、木の葉のかげにでもかくれたでしょう。ところが夜だかは、ほんとうは鷹(たか)の兄弟でも親類でもありませんでした。かえって、よだかは、あの美しいかわせみや、鳥の中の宝石のような蜂(はち)すずめの兄さんでした。蜂すずめは花の蜜(みつ)をたべ、かわせみはお魚を食べ、夜だかは羽虫をとってたべるのでした。それによだかには、するどい爪(つめ)もするどいくちばしもありませんでしたから、どんなに弱い鳥でも、よだかをこわがる筈(はず)はなかったのです。
\end{lstlisting}

を用意しておいて

\begin{lstlisting}
$ ./nan yodaka.txt
\end{lstlisting}

とコマンドを叩くと、

\begin{lstlisting}[caption=出力]
nan - novel analyzer 0.1

;; reulst
; info
papers: 1.85
chars: 673
lines: 9
n/w: 592/81
kj/h/k/o: 74/505/2/92

; warn
line-head-indent: 0
close-paren: 2
!?blank: 0
ellipsis: 0
dash: 0
\end{lstlisting}

と、解析結果を表示してくれます。
要するにノベルチェッカー\footnote{ノベルチェッカー簡易版, http://htmldwarf.hanameiro.net/tools/novelchecker\_easy.cgi}のコマンドライン版みたいなものです。
でも\textbf{完全に同じ動作はしません}。

nan は UTF-8、EUC-JP、SJIS (CP932) の三種類の文字コードを扱うことができます。
改行コード三種もばっちり扱えます。


\section{導入}
きっと Windows と Linux (Ubuntu) 向けにバイナリ用意します。


\section{使い方}
nan コマンドの使い方を見ていきます。

基本的な使い方は、コマンド名の後ろにファイル名を続けるだけです。

\begin{lstlisting}[caption=コマンドの使い方1]
$ nan yodaka.txt
\end{lstlisting}

もし問題のあった箇所に何が書いてあったのかを見たい場合、d オプションを使い前後の文字を表示させることができます。

\begin{lstlisting}[caption=コマンドの使い方2]
$ nan -d yodaka.txt
\end{lstlisting}

その他のオプションは、解析結果を別のプログラムで利用する場合とかに使えますが、小難しいので後述します。


\subsection{結果の見方}
冒頭で示した出力結果で nan の解析結果の見方を説明します。

\begin{lstlisting}[caption=出力]
nan - novel analyzer 0.1

;; reulst
; info
papers: 1.85
chars: 673
lines: 9
n/w: 592/81
kj/h/k/o: 74/505/2/92

; warn
line-head-indent: 0
close-paren: 2
!?blank: 0
ellipsis: 0
dash: 0
\end{lstlisting}

セミコロン「;」で始まっていない行が解析結果です。

\begin{itemize}
  \item papers \\
    400字詰め原稿用紙 (20x20) 換算の原稿用紙枚数です。
    行の途中で改行が入ってもちゃんと適当に計算します。

  \item chars \\
    総文字数です。

  \item lines \\
    総行数です。
    (なんだか1行数字が多いような……)

  \item n/w \\
    地の文/台詞の文字数です。
    地の文が n で、台詞が w です。 \\
    d オプションで地の文/台詞の割合をパーセントで追加表示します。
    割合は n/w\%: と表示されます。
    
  \item kj/h/k/o \\
    文中の各文字種 (漢字/ひらがな/カタカナ/その他文字) の文字数です。
    漢字が kj、ひらがなが h、カタカナが k、その他文字が o です。 \\
    d オプションで文字種の割合をパーセントで追加表示します。
    割合は kj/h/k/o\%: と表示されます。

  \item line-head-indent \\
    文中で冒頭字下げがされていない (行頭に全角スペースがない) 箇所の個数です。 \\
    d オプションで問題のあった箇所の前後 15 文字を追加表示します。

  \item close-paren \\
    閉じ括弧の後ろに句読点が置かれている箇所の個数です。
    サンプルの文章の台詞のような書き方をすると引っかかります。
    (僕が宮澤賢治方式が嫌いなわけではありません) \\
    d オプションで問題のあった箇所の前後 15 文字を追加表示します。

  \item !?blank \\
    感嘆符「!」や疑問符「?」の直後に全角スペースがついていない箇所の個数です。 \\
    d オプションで問題のあった箇所の前後 15 文字を追加表示します。

  \item ellipsis \\
    三点リーダ「…」が二連続で使われていない箇所の個数です。 \\
    d オプションで問題のあった箇所の前後 15 文字を追加表示します。

  \item dash \\
    ダッシュ「―」が二連続で使われていない箇所の個数です。 \\
    d オプションで問題のあった箇所の前後 15 文字を追加表示します。

\end{itemize}

ちなみに、d オプションをつけて解析するとこんな漢字の結果が出力されます。

\begin{lstlisting}[caption=出力 (dオプション付き)]
;; reulst
; info
papers: 1.80
chars: 673
lines: 9
n/w: 592/81
n/w%: 88.0/12.0
kj/h/k/o: 74/505/2/92
kj/h/k/o%: 11.0/75.0/0.3/13.7

; warn
line-head-indent: 0
close-paren: 2
  ** 1
    うに、鳥の仲間のつらよごしだよ。」
「ね、まあ、あのくちのお

  ** 2
    と、かえるの親類か何かなんだよ。」
 こんな調子です。おお、

!?blank: 0
ellipsis: 0
dash: 0
\end{lstlisting}



\subsection{オプション}



\end{document}
